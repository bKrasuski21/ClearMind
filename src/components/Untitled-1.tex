\documentclass[twoside,11pt]{article}

\usepackage{blindtext}

% Any additional packages needed should be included after jmlr2e.
% Note that jmlr2e.sty includes epsfig, amssymb, natbib and graphicx,
% and defines many common macros, such as 'proof' and 'example'.
%
% It also sets the bibliographystyle to plainnat; for more information on
% natbib citation styles, see the natbib documentation, a copy of which
% is archived at http://www.jmlr.org/format/natbib.pdf

% Available options for package jmlr2e are:
%
%   - abbrvbib : use abbrvnat for the bibliography style
%   - nohyperref : do not load the hyperref package
%   - preprint : remove JMLR specific information from the template,
%         useful for example for posting to preprint servers.
%
% Example of using the package with custom options:
%
% \usepackage[abbrvbib, preprint]{jmlr2e}

\usepackage{jmlr2e}

% Definitions of handy macros can go here

\newcommand{\dataset}{{\cal D}}
\newcommand{\fracpartial}[2]{\frac{\partial #1}{\partial  #2}}

% Heading arguments are {volume}{year}{pages}{date submitted}{date published}{paper id}{author-full-names}

\usepackage{lastpage}

% Short headings should be running head and authors last names


\firstpageno{1}

\begin{document}

\title{ClearMind: Migraine prediction powered by a deep neural network}

\author{\name Bernard Krasuski \email bernard.krasuski@ufl.edu \\
       \addr Herbert Wertheim College of Engineering\\
       University of Florida\\
       Gainesville, Fl 98195-4322, USA
      }

\editor{Aashish Dhawan?}

\maketitle

\begin{abstract}%   <- trailing '%' for backward compatibility of .sty file
Migraines are the sixth most disabling illness in the world. Migraines severely lower the quality of life of afflicted patients, and yet there is little to no research in prevention through prediction. A patient suffering from migraines can significantly reduce their symptoms through "environmental cleanup;" however, it tends to be rather confusing and require the input of a neurologist professional. This project was developed to alleviate the burdens associated with consulting with a medical professional. Utilizing the complex pattern recognition of deep neural networks, this project predicts the occurrence and triggers for migraines in individual patients. This is project is unique in the sense that it allows patients to add their own data to train the the deep neural network with. This gives incredibly personalized analysis, only paralleled by a medical professional. The generalized deep neural network model was trained on data accumulated from an online public database converted into a usable format. (SMOTE) Synthetic Minority Over-Sampling Technique, a data augmentation technique, was used on the database to improve results. The generalized deep neural network achieved over 99 percent accuracy for migraine prediction. In order to provide a platform for users to input data into the personalized deep neural network a web application was developed. This web application acts as both a calendar as well as a migraine data form, where users can input important factors such as sleep, stress, etc.

\end{abstract}

\begin{keywords}
KEYWORDS: Migraines; Predicting Migraines; Migraine Triggers; Machine Learning; SMOTE; Deep Neural Network.
\end{keywords}

\section{Introduction}

\blindmathpaper

Here is a citation \cite{chow:68}.

% Acknowledgements and Disclosure of Funding should go at the end, before appendices and references

\acks{All acknowledgements go at the end of the paper before appendices and references.
Moreover, you are required to declare funding (financial activities supporting the
submitted work) and competing interests (related financial activities outside the submitted work).
More information about this disclosure can be found on the JMLR website.}

% Manual newpage inserted to improve layout of sample file - not
% needed in general before appendices/bibliography.

\newpage

\appendix
\section{}
\label{app:theorem}

% Note: in this sample, the section number is hard-coded in. Following
% proper LaTeX conventions, it should properly be coded as a reference:

%In this appendix we prove the following theorem from
%Section~\ref{sec:textree-generalization}:

In this appendix we prove the following theorem from
Section~6.2:

\noindent
{\bf Theorem} {\it Let $u,v,w$ be discrete variables such that $v, w$ do
not co-occur with $u$ (i.e., $u\neq0\;\Rightarrow \;v=w=0$ in a given
dataset $\dataset$). Let $N_{v0},N_{w0}$ be the number of data points for
which $v=0, w=0$ respectively, and let $I_{uv},I_{uw}$ be the
respective empirical mutual information values based on the sample
$\dataset$. Then
\[
	N_{v0} \;>\; N_{w0}\;\;\Rightarrow\;\;I_{uv} \;\leq\;I_{uw}
\]
with equality only if $u$ is identically 0.} \hfill\BlackBox

\section{}

\noindent
{\bf Proof}. We use the notation:
\[
P_v(i) \;=\;\frac{N_v^i}{N},\;\;\;i \neq 0;\;\;\;
P_{v0}\;\equiv\;P_v(0)\; = \;1 - \sum_{i\neq 0}P_v(i).
\]
These values represent the (empirical) probabilities of $v$
taking value $i\neq 0$ and 0 respectively.  Entropies will be denoted
by $H$. We aim to show that $\fracpartial{I_{uv}}{P_{v0}} < 0$....\\

{\noindent \em Remainder omitted in this sample. See http://www.jmlr.org/papers/ for full paper.}


\vskip 0.2in
\bibliography{sample}

\end{document}
